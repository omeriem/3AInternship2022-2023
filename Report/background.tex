
\subsection{Model Driven Engineering}

\subsection*{A Methodology Based on Model-Driven Engineering for IoT Application Development}
Model driven engineering~\cite{•} eases the development of IoT Application. MDE allows model transformation in order to generate the code of the software applications for IoT. IoT systems must be physically or virtually interconnected and Service Oriented Architecture allows the deployment of such applications. SOA-based architecture is composed as 4 layers :
\begin{itemize}
    \item 
An Object Layer, which allows the identification, state management and data exchange of objects.
    \item
A Network Layer, which allows connection and communication between objects.
    \item A Service Layer, which allows management and creation of services used by users or softwares applications.
    \item
A Application Layer, which is responsible for delivering the applications to IoT users.

\end{itemize}

The paper given in~\cite{•} introduces a methodology for the development of software applications for IoT that is composed of 4 phases : 
\begin{itemize}
    \item 
Analysis of business requirements
\item
Definition of the business logic
\item
Design of the integrated services solution
\item
Generation of the technological solution
\end{itemize}
This methodology reduces the time and costs in software development by implementing automatic and semi-automatic model transformations.

\subsection*{Design of a domain specific language and IDE for Internet of things applications}
The paper presented in~\cite{•} offers a solution to the complexity of designing IoT applications with a wide range of wireless sensory networks (WSNs), devices, communication media, protocols and operating systems. In that regard, DSL-4-IoT Editor-Designer has been developed. The solution use : 
\begin{itemize}
    \item 
PervML,  a DSL that allow developers to describe pervasive system in a technology independent way 
\item
DiaSuite, a tool suite providing DSL designing tools, Java code generation, 2D renderer and deployment framework. 
\item
Node-RED, a visual tool allowing complex wirings and connections for IoT applications and objects.
\item
OpenHAB, a general-purpose framework for smart home.
\end{itemize}
And allows its users to build DSLs with a high level visual programming language built in JS.

\subsection*{MDE4IoT: Supporting the Internet of Things with Model-Driven Engineering}
IoT applications~\cite{•} brings many challenges for developers, among them , supporting complexity and heterogeneity management, supporting collaborative development, maximizing reusability of design artifacts, and providing self-adaptation of IoT systems. MDE4Iot is a framework allowing  high-level abstraction and separation of concerns to manage heterogeneity and complexity of Things, enabling collaborative development, enforcing reusability of design artifacts, and automation, in terms of model manipulations, enabling runtime self-adaptation. 

\subsection*{Model Driven Development for Internet of Things Application Prototyping}
This work given ~\cite{•} discusses the use of MDE in the development of IoT Applications, and provides a different viewpoint on the conception of such applications by placing domain modeling and object virtualization in the center of the architecture.

\subsection{Domain Specification Language: ECG et EEG}

Electrocardiograms (ECG) and Electroencephalograms (EEG) are processes that measure the heart and brain activities respectively using the electrical signals they produce. The correct modeling and treatment of these signals is essential for medical purposes because they are critical for the diagnosis and treatment of diseases. During the following sections, several applications for the treatment and modeling of these signals are going to be shown.

\subsubsection{Changes in Cortical Activity During Real and Imagined Movements: an ERP Studies}

Changes in cortical activities in movement imagination MI (when a person imagines performing a movement) and movement execution ME(when a person actually performs a movement) are similar. There is a similar topology for the activated brain areas for both ME and MI, especially in the collateral motor area of the brain. However contrary to ME, the activation of the motor neurons is insufficient for triggering spinal motor neurons, and therefore the movement is blocked.

Therefore, MI is very important for the use of prosthetic implants as it activates the motor areas of the brain even when it does not perform an action. Understanding the brain behavior during the MI is important, for this purpose electroencephalogram (EEG) is used, with the intent of recording the brain’s electrical activity and describing it through complex waves.  However, the signals are not-periodical, don't have standardized patterns, and have small voltage amplitudes, which make them easy to be mixed up with noise from exterior factors like heartbeat, muscle movements, and environment. In order to solve this, window functions are used to reduce noise and enhance movement intention signals.

\subsubsection{Window functions}

Window functions are used in signal processing for reducing noise and enhancing the original signal. There are several types of functions, each of them serving different types of signal waves.  For this reason, two studies were made to compare the performance of different window functions over EEG signals.\\

In \textit{Window Function for EEG Power Density Estimation and Its Application in SSVEP Based BCIs} the Rectangle, Hamming, Hann and Triangular functions were compared and the triangular function was the most accurate. Moreover, it was found that the use of a high pass filter enhances the accuracy of the signal produced.\\

On the other hand, in Window Functions Analysis in Filters for EEG Movement Intention Signals Bartlett, Blackman, Hanning, Hamming, Kaiser, Rectangular and Triangular window functions were compared, calculating the euclidean distance of the signal to the simulated signal. It was found that Blackman and Barlett window functions have the highest accuracy describe these signals.

\subsubsection{Formalizing electrocardiogram (ECG) signal behavior in event-B}

As the electrocardiogram is a process for measuring the heart's electrical signals, it is used for medical examinations to evaluate several medical conditions. This makes the treatment of ECG signals critical in situations like a stroke, where the medical response needs to be urgent. For this reason, its treatment needs to be done in real-time. This study implements a real-time monitoring system, using LabView, a medical laboratory toolkit, to treat the heartbeat signals by reducing noise, and when an anomaly is found an SMS is sent to a designated doctor.

\subsubsection{Development of real-time ECG signal monitoring system for telemedicine application}
There is a lack of formal specification of the interpretation of these signals. This study proposes a model in Event-B, a language for specifying and verifying complex systems using proof. The three stages in which a heartbeat produces signals depolarization were modeled, namely atrial depolarization, ventricular depolarization, and ventricular repolarization, and among them three characteristic waves observable by the ECG P, QRS and T. It was constructed using invariants of the heart behavior, which could help identify when a problem is occurring.


\subsection{Artificial Intelligence} 
Artificial intelligence has been an interesting domain in this era, allowing developers and others to benefit from its capacity. Although this domain is usually vast, its current branch called Machine Learning(ML) has seen a rise in popularity. Machine learning uses statistical modelling and computational learning technology in order to recognise pattern in a given datasets. All the ML existing techniques can be divided into several categories based on its different characteristic: frequency of learning, nature of learning, etc.

For the frequency of learning, we can divide the ML techniques into two categories, online and offline learning. Online Learning uses real time input and output to incrementally modify its state for each observation. However in an offline learning, a batch of this observations are sent to the model from time to time.

Nonetheless, ML techniques can also be recognized from its different nature of learning. In this category, the ML techniques are called supervised and unsupervised learning. A supervised learning differs from unsupervised learning in terms of the pre-recognition of the labels or classed that can be found in the datasets, such as regression or classification. The unsupervised techniques starts the learning process with zero knowledge of the data, constructing the labels in the dataset one step after another, where the clustering technique is the mostly wide used. 

The challenges in using Artificial Intelligence(or more precisely Machine Learning) in Model-Driven Engineering is due to the lack of datasets that the ML models can learn from. Even if the datasets are available, the heterogenous nature of the datasets specifically for the Internet of Things(IoT) technologies are delimiting the application of AI techniques in the domain. The learning granularity of the AI models needs to be fine-grained in order to obtain a more accurate and efficient learning.

The different notions of models in AI and MDE also brings some difficulty in integrating these two domain together. ML models or techniques as mentioned above are completely distinct than the models used in MDE; one is a statistical model while another is a structural model. Various works such as ML-Quadrat are looking forward to create a synergy between this models, allowing a more seamless model engineering in the domain.

\subsection{IoT Definition : }
%Do not forget that we are particularly working on IoT domain application and specifically on signal synchronisation for healthcare use case (EEG and ECG).
  \begin{itemize}
  
      \item IOT stands for Internet Of Things, it can be defined as " a self configuring and adaptive system consisting of networks of sensors and smart objects whose purpose is to interconnect ‘all’ things" according to the IEEE Iot Community. In the article \uline{Building Caring Healthcare Systems in the Internet of Things}, three general classes for use cases of IOT in healthcare are defined. In this domain, Iot is mainly used to collect data for these purposes : Tracking humans, tracking things and tracking both.

      \item One important aspect of Iot, especially for healthcare Iots, is clock synchronisation. In the article \uline{A System for Clock Synchronization in an Internet of Things}, a solution is proposed for this challenge. This solution is SPoT, a packet exchange protocol, and it has advantages in comparison to other standard protocols.

      \item IOT applications\newline
      \begin{itemize}
        \item The article \uline{Building Caring Healthcare Systems
        in the Internet of Things} shows a way to describe, specify and implement an IOT application. It underlines the need for an expertise in healthcare domain and         a strong collaboration between engineers and healthcare workers.
        \item One exemple of an Iot implementation is shown in the article : \uline{Implementation of a portable device for real-time ECG signal analysis}. In this             article, this Iot application for real-time
        electrocardiogram (ECG) acquisition is described.
        \item Finally, the article \uline{Create Your Own Internet of Things} targets anyone who wants to make their own iot device by proposing a selection of           hardwares, platforms, and programming langages for Iot creation. It compares different aspects such as prices, connectivity or compatibility.
      \end{itemize}
  \end{itemize}
